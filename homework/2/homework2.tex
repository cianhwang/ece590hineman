\documentclass{article}

\usepackage{fullpage}
\usepackage{hyperref}
\hypersetup{
  colorlinks = true
}
\usepackage{listings}
\lstset{basicstyle=\ttfamily\footnotesize,breaklines=true}


\title{Homework 2---ECE590--001---{\bf Due: 19 Feb 2020}}
\date{2/29/2020}
\begin{document}
\maketitle
\begin{enumerate}
\item Implement gridworld using OpenAI gym specification. You will implement a
sub-class of {\tt Env} called {\tt gridworld}.
  \begin{enumerate}
  \item Your class should allow for arbitrary rectangular grids be able to
    specify arbitrary reward per user specification. You will also need a way to
    pass a done condition.
  \item Your class should be initialized as follows:
    \begin{lstlisting}[language=python]
      my_grid = gridworld(reward=my_reward) # where my_reward is a numpy.array of shape (n,m)
    \end{lstlisting}
  \item Your class needs to meet the specifications set down in docstring for
    {\tt Env}: \url{https://github.com/openai/gym/blob/master/gym/core.py}.
    Implement the API methods; all except render are required for this assignment.
    There will be a bonus for render.
  \item Check that your class is working and meets specification by running the
    {\tt gym} {\em random agent}:
    \url{https://github.com/openai/gym/blob/master/examples/agents/random_agent.py}.
  \item Demonstrate class can encode the gridworld Example 3.5 on page 60 of
    \href{http://incompleteideas.net/book/RLbook2018.pdf}{Sutton and Barto}.
  \end{enumerate}
\item
  Sample the dynamics of your gridworld under a random policy and estimate
  state-value function like in
  \href{http://www.incompleteideas.net/book/RLbook2018.pdf}{Example 3.8 page 65 of Sutton and Barto}.
  \begin{enumerate}
  \item Make a function from the rollout/trajectory code
    \url{https://github.com/openai/gym/blob/master/examples/agents/random_agent.py}.
  \item Call this function in such a way that you can estimate the state-value
    function for the random policy and the gridworld defined originally in Example
    3.5
  \end{enumerate}
\item Train a policy for the Example 3.5 gridworld using the Vanilla Policy
  Gradient (VPG) code provided in spinning up
  \url{https://spinningup.openai.com/en/latest/algorithms/vpg.html}.
  \begin{enumerate}
  \item Set a trajectory length so that 20 states are visited in each
    trajectory. The OpenAI gym framework this is essentially when you require {\tt
      step} to return the {\tt done} flag.
  \item Read and execute the Vanilla policy gradient code on your {\tt
      gridworld} environment:
    \url{https://spinningup.openai.com/en/latest/algorithms/vpg.html}. Note that
    this code despite being called {\em vanilla} already includes {\em Generalized
      Advantage Estimation}. The more basic code is here
    \url{https://github.com/openai/spinningup/blob/master/spinup/examples/pg_math/1_simple_pg.py}
  \item Supply a training artifact (a plot of either timesteps vs. total reward,
    trajectories sampled vs. total reward) showing that VPG converges to some score.
    In words, write down an explanation for this score in terms of the trajectory
    length and reward specified in Example 3.5.
  \item The VPG code estimates the state-value function. Plot these values
    (seaborn should do this nicely).
  \end{enumerate}
\end{enumerate}
\end{document}
